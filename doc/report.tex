\documentclass[a4paper]{article}

\usepackage{hyperref}
\usepackage{graphicx}
\usepackage{amsmath}
\usepackage{amsthm}
\usepackage{amsfonts}

\newtheorem{algorithm_def}{Algorithm}[section]
\newtheorem{definition}{Definition}[section]
\newtheorem{lemma}{Lemma}[section]
\newtheorem{theorem}{Theorem}[section]
\newtheorem{corollary}{Corollary}[section]

\DeclareMathOperator*{\argmin}{arg\,min}
\DeclareMathOperator*{\argmax}{arg\,max}

%opening
\title{Robot on icy surface \\ {\small CS6244 Robot Motion Planning \& Control (Prof David Hsu)} }
\author{
  Jahja, Irvan\\
  {\small A0123879R} \\ \texttt{dolphinigle.mailbox@gmail.com}
  \and
  Nguyen, Rang H. M.\\
  {\small A0123879R} \\ \texttt{dolphinigle.mailbox@gmail.com}
}

\begin{document}

\maketitle

\section{Motivation}
% * Provide an intuitive description of the problem domain. Describe the motivation and potential applications.
A car-like robot was dropped on the northern pole few days ago. It has collected various
data on the north pole, and now is ready for pickup at a specific destination,
and with a specific orientation. However, due to global warming, pool of water
forms on the icy surface, and falling into any of them spells certain disfunction
for the robot. Furthermore, icy surface make it very hard for the robot to
follow paths exactly, rendering normal techniques to solve the non-holonomic
motion planning less useful. How are we going to save the robot?

Enter this project. Our algorithm is able to guide the robot to its destination
reliable and, if the robot slips, correct its path, all in real-time. Our algorithm
is able to guide the robot despite uncertainty in its movement -- the preceeding
paragraph gives one such surface that may have unignorable uncertainty.

This project aims to devise such an algorithm, and demonstrates exactly why
we are not allowed to ignore uncertainty in the experiments.

\section{Problem statement}
% * State the technical problem formally. 

\subsection{Configuration space}
We consider a point robot moving on a two dimensional space $\mathbb{R}^2$.
The robot position is denoted by its coordinates $(x, y)$. We restrict the space
to $[0\ldots1, 0\ldots1]$ (that is, we must have $0 \le x \le 1$ and $0 \le y \le 1$).
The robot also has an orientation, denoted by $\theta$. Thus, the robot's
configuration space is $(x, y, \theta)$ of dimension $3$.

In addition, there are polygonal obstacles lying on the space. The robot
are not allowed to collide with these obstacles -- doing so results in the
failure of the execution. In addition, the robot is also not allowed to go
past the boundaries of the space.

We assume that the robot is able deduce its current configuration space at
any time -- that is, we assume perfect sensing and knowledge of the configuration space
and the configuration of the robot.

\subsection{Robot movement and non-holonomic constraint}
The robot's movement is governed by a car-like non-holonomic law -- the robot
has a length of $L$ and has a maximum steering distance of $\theta_{\text{max}}$. When the
robot's steering angle is at $\alpha$, then the turning radius of the robot
is $L / \sin(\alpha)$, meaning that if the robot steers it with angle of $\alpha$,
it will roughly follow an arc of a circle of radius $L / \sin(\alpha)$.

For simplicity, we assume that the robot do not have the capabilities to move
backwards. Extending the algorithm to support this movement is immediate, but
does not give interesting results.

\subsection{Uncertainty}
In addition, the robot is unable to execute commands perfectly. The coordinates
and orientation of the robot suffers from a noise sampled from an unknown distribution
with known variance (in practice, we can probably measure this variance).
Thus, when we perform integration over the length of the path, by the central
limit theorem, this value is approximately normal. Thus, we model the noise
using normal distribution with variance $\sigma^2$ for the coordinates and
$\sigma_\theta^2$ for the orientation.

\subsection{Objective}
The objective of the robot is to approximately reach a specific location with a
specific orientation. More exactly, the robot is given a goal configuration
$(x_g, y_g, \theta_g)$, and the robot must reach a configuration $(x, y, \theta)$
such that $|\theta - \theta_g| < \delta$ and $|(x, y)^2 - (x_g, y_g)^2| < T$,
for a threshold $T$ and orientation tolerance $\delta$. Note that the probability
of the robot reaching exactly the configuration $(x_g, y_g, \theta_g)$ might be
zero due to uncertainty in movement.

\section{Algorithm description}
% * Present your algorithm. Describe the input and output if necessary. 
We model the problem using the Markov Decision Process.

\subsection{Discretization}

\subsection{Markov model}

\section{Results and discussions}
% * Present the theoretical analysis, experiment results, comparison, etc., as applicable.

\subsection{Effect of axis resolution}

\subsection{Effect of variance}

\subsection{Effect of reward}

\subsection{Effect of discount factor}


\section{Conclusion}
% * Briefly summarize your main findings.

\end{document}

